% \documentclass{beamer}
% \usetheme{Szeged}

% \begin{document}


%-------------------------------------------------------------------------------
%							FIRST SECTION
%-------------------------------------------------------------------------------

%------------ Intro Part 2
% - le site de l'IRMA est riche et complet
% - De M.NAVORET et FRANCK vers l'equipe MOCO (leurs travaux, etc..):
% - l'equipe probabilite

\section{Introduction}    % L'IRMA

\subsection{L'IRMA}
  
\begin{frame}
\frametitle{L'équipe MOCO}
% MM. Franck et Navoret
	Plusieurs membres parmi lesquels MM. : % M. Prud'homme aussi
	\begin{itemize}
		\item Emmanuel FRANCK % Dernier expose: Base de modèles épidémiologiques, covid et contrôle (2020)
		\item Laurent NAVORET % Dernier expose: Modèle macroscopique pour un système de particules discoïdales en interactions d'alignement (2015)
  \end{itemize}
  Responsables des séminaires en EDP

  \pause
  $\newline$
  \only<2->{Des activités diverses :}
  % L'equipe MOCO en general (analyse des EDP, de la théorie du contrôle, du calcul scientifique et haute performance, et des statistiques.)
	\begin{itemize}[<+>]
		\item Partenariats internationaux % (Portugal, Allemagne, USA, etc.)  % Projets (Examag Spexxa, MAToS, projet EUROFUSION)
		\item Partenariats industriels  % 
		\item Modélisation des plasmas  % L’équipe projet INRIA TONUS qui lui est adossee
  \end{itemize}

\end{frame}

% \subsection{L'équipe Probabilités}
	
\begin{frame}
\frametitle{L'équipe Probabilités}
% MM. Franck et Navoret
	Plusieurs membres parmi lesquels M. : % M. Prud'homme aussi
	\begin{itemize}
		\item Vincent VIGON
  \end{itemize}

  \pause
  % L'equipe Probabilites en general
  $\newline$
  \only<2->{Des activités diverses :}
	\begin{itemize}[<+>]
		\item Partenariats internationaux %(Allemagne, Australie, Chine, etc)  % Actuariat, Transport optimal, Matrrice Aleatoire
		\item Séminaire (de calcul) stochastique  % 
  \end{itemize}

% You get the point, ce sint de grosses equipes de recherches tres actives! Et des que j'ai vu qu'elles allait encadrer le projet, j'ai saute sur l'occasion

\end{frame}

\subsection{Le sujet du stage}

\begin{frame}
  \frametitle{Le(s) problème(s) à résoudre}

  \pause

\begin{columns}
 \begin{column}{0.5\textwidth}
  \centering
    Problème direct \\ (\scriptsize Résolution de l'ETR par un schéma de "splitting")
    % Image de densite -> signal sur les bords
      % \includegraphics[width=5cm]{ProblemeDirect}       
  \end{column}

  \pause

 \begin{column}{0.5\textwidth}
    \centering
    Problème inverse \\ (\scriptsize Reconstruction de la densité par un réseau de neurones)
    %Image de signal sur les bords -> densite
      % \includegraphics[width=5cm]{ProblemeInverse}       
 \end{column}
\end{columns}

\begin{figure}
  \includegraphics<1->[width=4cm]{PBInverse}         
\end{figure}

\end{frame}
%-------- Vrai debut de l'introduction (PB INVERSE)
\begin{frame}
  \frametitle{Les points pour situer le stage}

  \begin{enumerate}[<+>]
    \item Explosion du Deep Learning % Depuis le debut de la decenie 2010, le Machine Learning a considerablement pris de l’ampleur (2015 a l’ILSVRC, etc..)
    %%%%%% IMAGE DU DEEP LEARNING
    \item Application du Machine Learning en imagerie médicale % Avant de soigner les cancers, on doit detecter les tumeurs sont plus denses que les tissus sains (Chercher d'autres applications)
    %%%%%% IMAGE DU MEDICAL
    \item Réévaluation des méthodes de résolution de problèmes inverse % Les problemes inverses sont difficiles. ... Les algo d'optimisation classiques marchent tres bien. En fait on s'est referer aux travaux de Maya et Guillaume Dolle. L'avantage que peuvent offrir les ANN c'est juste la simplicite, et la rapidite, et une generalisation (non specificite aux probleme)
    %%%%%% IMAGE DU PB INVERSE
  \end{enumerate}
  
\end{frame}

% \end{document}
