%-------------------------------------------------------------------------------
%							PREAMBULE
%-------------------------------------------------------------------------------

\documentclass[xcolor=dvipsnames]{beamer} % dvipsnames gives more built-in colors

\usetheme{Szeged}
\useoutertheme{miniframes} % Alternatively: miniframes, infolines, split, smoothbars, smoothtree, shadow
\useinnertheme{rectangles}

\usepackage{media9}
\usepackage{graphicx}

\usepackage{docmute} % To include multiple files

\usepackage{tikz}
\usetikzlibrary{positioning,shapes,arrows}
\usetikzlibrary{babel}      % Utiliser ce Babel pour eviter les problemes avec les animations TIKX
% \usepackage{csquotes}

\usepackage{iwona}
% \usepackage{lmodern}
% \usepackage[french]{babel}
\graphicspath{ {./Figures/} }
\usepackage[utf8]{inputenc}
\newcommand{\bvec}[1]{\textbf{#1}} 
\usepackage{amsmath,bm,mathtools}
% \usepackage{unicode-math}
\usepackage{siunitx}

\usepackage[backend=bibtex,style=authoryear,maxnames=2,natbib=true]{biblatex} % Use the bibtex backend with the authoryear citation style (which resembles APA)
\addbibresource{bibliographie.bib} % The filename of the bibliography
\usepackage[autostyle=true]{csquotes} % Required to generate language-dependent quotes in the bibliography 
\renewcommand*{\bibfont}{\tiny} % Pour reduire la taille des references

\usepackage[font=scriptsize]{caption}
\captionsetup{labelformat=empty,labelsep=none}    % Pour retirer le terme figure des titres

\usepackage{booktabs}

\definecolor{UBCblue}{rgb}{0.04706, 0.13725, 0.26667} % UBC Blue (primary)

\definecolor{MyGreen}{RGB}{0, 50, 0}
\definecolor{MyRed}{RGB}{100, 0, 0}

% \usecolortheme[named=UBCblue]{structure}
\usecolortheme[named=MyGreen]{structure} % Sample dvipsnames color
\setbeamercolor{alerted text}{fg=MyRed}
% \definecolor{darkred}{rgb}{0.4,0.5,0}
% \setbeamercolor{titlelike}{parent=structure,bg=yellow!85!orange}

\setbeamercovered{transparent}    %% Pour avoir de la tranparence

%-------------------------------------------------------------------------------
%							A CUSTOM FOOTLINE
%-------------------------------------------------------------------------------

\makeatletter
\setbeamertemplate{footline}{
	\begin{beamercolorbox}[colsep=1.5pt]{upper separation line foot}
	\end{beamercolorbox}
  \leavevmode%
  \hbox{%
  \begin{beamercolorbox}[wd=.333333\paperwidth,ht=2.25ex,dp=1ex,center]{author in head/foot}%
	% \usebeamerfont{author in head/foot}\insertshortauthor\expandafter\beamer@ifempty\expandafter{\beamer@shortinstitute}{}{~~(\insertshortinstitute)}
	\insertshortauthor
  \end{beamercolorbox}%
  \begin{beamercolorbox}[wd=.333333\paperwidth,ht=2.25ex,dp=1ex,center]{title in head/foot}%
    \usebeamerfont{title in head/foot}\insertshorttitle
  \end{beamercolorbox}%
  \begin{beamercolorbox}[wd=.333333\paperwidth,ht=2.25ex,dp=1ex,right]{date in head/foot}%
    \usebeamerfont{date in head/foot}\insertshortdate{}\hspace*{2em}
    \insertframenumber{} / \inserttotalframenumber\hspace*{2ex} 
  \end{beamercolorbox}}%
  \vskip0pt%

  \begin{beamercolorbox}[colsep=1.5pt]{lower separation line foot}
  \end{beamercolorbox}

}
\makeatother

%-------------------------------------------------------------------------------
%							FIRST TITLE PAGE
%-------------------------------------------------------------------------------

\title[Problème Inverse : Transfert Radiatif et Apprentissage]{Problème Inverse : Transfert Radiatif et Apprentissage}
\date{\today}
\author[Roussel Desmond NZOYEM]{Roussel Desmond NZOYEM}

\institute[Université de Strasbourg]{Université de Strasbourg\\UFR de mathématiques et d'informatique\\Master 1 CSMI}

\begin{document}

\begingroup
\setbeamertemplate{navigation symbols}{}
\setbeamertemplate{headline}{\vspace{0.5cm}%
  \hspace*{0.8cm}%
  \includegraphics[width=3cm]{LogoUnistra}
  \hfill\raisebox{.2cm}{}\hfill%
  \includegraphics[width=2.2cm]{LogoIRMA}
  \hspace*{0.8cm}%
}
% \begin{frame}
% \maketitle
% \end{frame}
\endgroup

%-------------------------------------------------------------------------------
%							SECOND TITLE PAGE
%-------------------------------------------------------------------------------

%------------ Intro Part 1
% - Moi (Nom-prenom-classe)
% - Remerciement des profs, qui ont ausi encadrer le projer
% - Parler du projet

\begingroup  % A new griou whose informations are not cannon., just for convenience

\title[Problème Inverse : Transfert Radiatif et Apprentissage]{Simulation 2D de l’équation du transfert radiatif et reconstruction de la densité par un réseau de neurones}

\institute[Université de Strasbourg]{\small \textbf{ \hspace*{0.1mm} Enseignant référent} \hspace*{11mm} \textbf{Maîtres de stage} \\ \footnotesize Christophe PRUD'HOMME \hspace*{6mm} Emmanuel FRANCK \\ \hspace*{44.5mm} Laurent NAVORET \\ \hspace*{44.5mm} Vincent VIGON}

\author[Roussel Desmond NZOYEM]{\textbf{Stagiaire} \\ Roussel Desmond NZOYEM}
\date[\today]{\footnotesize Année académique 2019/2020}

\setbeamertemplate{navigation symbols}{}
\setbeamertemplate{headline}{\vspace{0.5cm}%
  \hspace*{0.8cm}%
  \includegraphics[width=3cm]{LogoUnistra}
  \hfill\raisebox{.2cm}{\normalsize Soutenance de stage}\hfill%
  \includegraphics[width=2.2cm]{LogoIRMA}
  \hspace*{0.8cm}%
}
\begin{frame}[fragile]
\maketitle
\end{frame}
\endgroup

%-------------------------------------------------------------------------------
%							INCLUDE THE CHAPTERS
%-------------------------------------------------------------------------------

% \documentclass{beamer}
% \usetheme{Szeged}

% \begin{document}


%-------------------------------------------------------------------------------
%							FIRST SECTION
%-------------------------------------------------------------------------------

%------------ Intro Part 2
% - le site de l'IRMA est riche et complet
% - De M.NAVORET et FRANCK vers l'equipe MOCO (leurs travaux, etc..):
% - l'equipe probabilite

\section{Introduction}    % L'IRMA

\subsection{L'IRMA}
  
\begin{frame}
\frametitle{L'équipe MOCO}
% MM. Franck et Navoret
	Plusieurs membres parmi lesquels MM. : % M. Prud'homme aussi
	\begin{itemize}
		\item Emmanuel FRANCK % Dernier expose: Base de modèles épidémiologiques, covid et contrôle (2020)
		\item Laurent NAVORET % Dernier expose: Modèle macroscopique pour un système de particules discoïdales en interactions d'alignement (2015)
  \end{itemize}
  Responsables des séminaires en EDP

  \pause
  $\newline$
  \only<2->{Des activités diverses :}
  % L'equipe MOCO en general (analyse des EDP, de la théorie du contrôle, du calcul scientifique et haute performance, et des statistiques.)
	\begin{itemize}[<+>]
		\item Partenariats internationaux % (Portugal, Allemagne, USA, etc.)  % Projets (Examag Spexxa, MAToS, projet EUROFUSION)
		\item Partenariats industriels  % 
		\item Modélisation des plasmas  % L’équipe projet INRIA TONUS qui lui est adossee
  \end{itemize}

\end{frame}

% \subsection{L'équipe Probabilités}
	
\begin{frame}
\frametitle{L'équipe Probabilités}
% MM. Franck et Navoret
	Plusieurs membres parmi lesquels M. : % M. Prud'homme aussi
	\begin{itemize}
		\item Vincent VIGON
  \end{itemize}

  \pause
  % L'equipe Probabilites en general
  $\newline$
  \only<2->{Des activités diverses :}
	\begin{itemize}[<+>]
		\item Partenariats internationaux %(Allemagne, Australie, Chine, etc)  % Actuariat, Transport optimal, Matrrice Aleatoire
		\item Séminaire (de calcul) stochastique  % 
  \end{itemize}

% You get the point, ce sint de grosses equipes de recherches tres actives! Et des que j'ai vu qu'elles allait encadrer le projet, j'ai saute sur l'occasion

\end{frame}

\subsection{Le sujet du stage}

\begin{frame}
  \frametitle{Le(s) problème(s) à résoudre}

  \pause

\begin{columns}
 \begin{column}{0.5\textwidth}
  \centering
    Problème direct \\ (\scriptsize Résolution de l'ETR par un schéma de "splitting")
    % Image de densite -> signal sur les bords
      % \includegraphics[width=5cm]{ProblemeDirect}       
  \end{column}

  \pause

 \begin{column}{0.5\textwidth}
    \centering
    Problème inverse \\ (\scriptsize Reconstruction de la densité par un réseau de neurones)
    %Image de signal sur les bords -> densite
      % \includegraphics[width=5cm]{ProblemeInverse}       
 \end{column}
\end{columns}

\begin{figure}
  \includegraphics<1->[width=4cm]{PBInverse}         
\end{figure}

\end{frame}
%-------- Vrai debut de l'introduction (PB INVERSE)
\begin{frame}
  \frametitle{Les points pour situer le stage}

  \begin{enumerate}[<+>]
    \item Explosion du Deep Learning % Depuis le debut de la decenie 2010, le Machine Learning a considerablement pris de l’ampleur (2015 a l’ILSVRC, etc..)
    %%%%%% IMAGE DU DEEP LEARNING
    \item Application du Machine Learning en imagerie médicale % Avant de soigner les cancers, on doit detecter les tumeurs sont plus denses que les tissus sains (Chercher d'autres applications)
    %%%%%% IMAGE DU MEDICAL
    \item Réévaluation des méthodes de résolution de problèmes inverse % Les problemes inverses sont difficiles. ... Les algo d'optimisation classiques marchent tres bien. En fait on s'est referer aux travaux de Maya et Guillaume Dolle. L'avantage que peuvent offrir les ANN c'est juste la simplicite, et la rapidite, et une generalisation (non specificite aux probleme)
    %%%%%% IMAGE DU PB INVERSE
  \end{enumerate}
  
\end{frame}

% \end{document}


\begin{frame}
  \scriptsize
  \frametitle{Sommaire}
  \tableofcontents
\end{frame}

\input{Sections/Section2}

% \documentclass{beamer}
% \usetheme{Szeged}

% \begin{document}

%-------------------------------------------------------------------------------
%							THIRD SECTION
%-------------------------------------------------------------------------------
\section{Le probleme en 1D}
Le probleme en 1D
\subsection{Simulation}

\subsection{Apprentissage}


% \end{document}


% \documentclass{beamer}
% \usetheme{Szeged}

% \begin{document}

%-------------------------------------------------------------------------------
%							FOURTH SECTION
%-------------------------------------------------------------------------------
\section{Resultats 2D}
Le probleme en 2D

\subsection{Simulation}

\subsection{Apprentissage}

% \end{document}


% \documentclass{beamer}
% \usetheme{Szeged}

% \begin{document}

%-------------------------------------------------------------------------------
%							FITH SECTION
%-------------------------------------------------------------------------------


\section{Conclusion}
% Conclusion

\subsection{Sur l'apprentissage}
\begin{frame}
    \frametitle{Bilan de l'apprentissage}
    \begin{enumerate}
        \item \textbf{Régression 1D} : Permet de détecter la hauteur du créneau %sur la densité d'un domaine 1D (avec la meilleure corrélation de tous les apprentissages). Elle n'a cependant pas été capable de détecter la position du créneau, probablement dû au caractère mal posé du problème inverse.
        \item \textbf{Classification 2D} : Permet de localiser l'ordonnée du créneau %en le situant par rapport aux sources sur la gauche d'un domaine 2D. En augmentant leur nombre et en plaçant certaines sources en haut (ou en bas) du domaine, on pourrait localiser avec plus de finesse l'abscisse et l'ordonnée du créneau.
        \item \textbf{Régression 2D} : Permet de prédire tous les attributs essentiels du créneau (abscisse, ordonnée, et hauteur)%, tout ceci avec une très forte précision (score personnalisé s'élevant à 93 \%). 
      \end{enumerate}
\end{frame}

\subsection{Generale}
\begin{frame}
    \frametitle{Bilan du stage}
    \begin{figure}
        \includegraphics[width=10cm]{MilestonesRoadmap}       
        \caption{Points tournants du le stage}
    \end{figure}
\end{frame}

\begin{frame}
    \frametitle{Apports et enseignements}
    \begin{itemize}
        \item Developpement C++ et Python   % Mettre en pratique les connaissances de CSMI
        \item Equations aux derivees partielles % technique de verification
        \item Reseaux de neurones % (Keras, Tensorflow, learning rate)
        \item Experience dans un milieu de recherche % J'ai apprecieer travailler sur IA+EDP
        % \item Point negatif: Manque de coordination (A cause du COVID)
    \end{itemize}
\end{frame}


% \end{document}


% %-------------------------------------------------------------------------------
% %							THANK YOU NOTE
% %-------------------------------------------------------------------------------

\begin{frame}
  \large
  \centering
  Merci pour votre attention
\end{frame}

% %-------------------------------------------------------------------------------
% %							THE BIBLIOGRAPHY
% %-------------------------------------------------------------------------------
\appendix   % Pour retirer les references de la bare de navigation
\vspace*{0.5mm}
\printbibliography


\end{document}
