%-------------------------------------------------------------------------------
%							PREAMBULE
%-------------------------------------------------------------------------------

\documentclass[xcolor=dvipsnames]{beamer} % dvipsnames gives more built-in colors

\usetheme{Szeged}
\useoutertheme{miniframes} % Alternatively: miniframes, infolines, split, smoothbars, smoothtree, shadow
\useinnertheme{rectangles}

% \usepackage{iwona}
\usepackage{lmodern}
\usepackage[french]{babel}
\graphicspath{ {./Figures/} }
\usepackage[utf8]{inputenc}
\newcommand{\bvec}[1]{\mathbf{#1}} 
\usepackage{amsmath,bm,mathtools}
\usepackage{siunitx}

\usepackage{docmute}


\definecolor{UBCblue}{rgb}{0.04706, 0.13725, 0.26667} % UBC Blue (primary)

\definecolor{MyGreen}{RGB}{0, 50, 0}

% \usecolortheme[named=UBCblue]{structure}
\usecolortheme[named=MyGreen]{structure} % Sample dvipsnames color

% \definecolor{darkred}{rgb}{0.4,0.5,0}
% \setbeamercolor{titlelike}{parent=structure,bg=yellow!85!orange}


%-------------------------------------------------------------------------------
%							A CUSTOM FOOTLINE
%-------------------------------------------------------------------------------

\makeatletter
\setbeamertemplate{footline}{
	\begin{beamercolorbox}[colsep=1.5pt]{upper separation line foot}
	\end{beamercolorbox}
  \leavevmode%
  \hbox{%
  \begin{beamercolorbox}[wd=.333333\paperwidth,ht=2.25ex,dp=1ex,center]{author in head/foot}%
	% \usebeamerfont{author in head/foot}\insertshortauthor\expandafter\beamer@ifempty\expandafter{\beamer@shortinstitute}{}{~~(\insertshortinstitute)}
	\insertshortauthor
  \end{beamercolorbox}%
  \begin{beamercolorbox}[wd=.333333\paperwidth,ht=2.25ex,dp=1ex,center]{title in head/foot}%
    \usebeamerfont{title in head/foot}\insertshorttitle
  \end{beamercolorbox}%
  \begin{beamercolorbox}[wd=.333333\paperwidth,ht=2.25ex,dp=1ex,right]{date in head/foot}%
    \usebeamerfont{date in head/foot}\insertshortdate{}\hspace*{2em}
    \insertframenumber{} / \inserttotalframenumber\hspace*{2ex} 
  \end{beamercolorbox}}%
  \vskip0pt%

  \begin{beamercolorbox}[colsep=1.5pt]{lower separation line foot}
  \end{beamercolorbox}

}
\makeatother

%-------------------------------------------------------------------------------
%							FIRST TITLE PAGE
%-------------------------------------------------------------------------------

\title[Problème Inverse : Transfert Radiatif et Apprentissage]{Problème Inverse : Transfert Radiatif et Apprentissage}
\date{\today}
\author[Roussel Desmond NZOYEM]{Roussel Desmond NZOYEM}

\institute[Université de Strasbourg]{Université de Strasbourg\\UFR de mathématiques et d'informatque\\Master 1 CSMI}

\begin{document}

\begingroup
\setbeamertemplate{navigation symbols}{}
\setbeamertemplate{headline}{\vspace{0.5cm}%
  \hspace*{0.8cm}%
  \includegraphics[width=3cm]{LogoUnistra}
  \hfill\raisebox{.2cm}{\normalsize Soutenance de stage}\hfill%
  \includegraphics[width=2.2cm]{LogoIRMA}
  \hspace*{0.8cm}%
}
\begin{frame}
\maketitle
\end{frame}
\endgroup

%-------------------------------------------------------------------------------
%							SECOND TITLE PAGE
%-------------------------------------------------------------------------------

%------------ Intro Part 1
% - Moi (Nom-prenom-classe)
% - Remerciement des profs, qui ont ausi encadrer le projer
% - Parler du projet

\begingroup  % A new griou whose informations are not cannon., just for convenience

\title[Problème Inverse : Transfert Radiatif et Apprentissage]{Simulation 2D de l’équation du transfert radiatif et reconstruction de la densité par un réseau de neurones}

\author[Roussel Desmond NZOYEM]{Roussel Desmond NZOYEM\\[2mm]{\small \textbf{ \hspace*{0.1mm} Ensignant referent} \hspace*{11mm} \textbf{Maitre de stage} \\ \footnotesize Christophe PRUD'HOMME \hspace*{6mm} Emmanuel FRANCK \\ \hspace*{44.5mm} Laurent NAVORET \\ \hspace*{44.5mm} Vincent VIGON}}

\institute[Université de Strasbourg]{}
\date[\today]{\footnotesize Annee Academique 2019/2020}

\setbeamertemplate{navigation symbols}{}
\setbeamertemplate{headline}{\vspace{0.5cm}%
  \hspace*{0.8cm}%
  \includegraphics[width=3cm]{LogoUnistra}
  \hfill\raisebox{.6cm}{}\hfill%
  \includegraphics[width=2.2cm]{LogoIRMA}
  \hspace*{0.8cm}%
}
\begin{frame}
\maketitle
\end{frame}
\endgroup

%-------------------------------------------------------------------------------
%							FIRST SECTION
%-------------------------------------------------------------------------------

%------------ Intro Part 2
% - le site de l'IRMA est riche et complet
% - De M.NAVORET et FRANCK vers l'equipe MOCO (leurs travaux, etc..):
% - l'equipe probabilite

\section{Introduction}    % L'IRMA

\subsection{L'IRMA}
	
\begin{frame}
  % MM. Franck et Navoret
	L'equipe MOCO compte plusieurs membres parmi lesquels MM.: % M. Prud'homme aussi
	\begin{itemize}
		\item Emmanuel FRANCK % Dernier expose: Base de modèles épidémiologiques, covid et contrôle (2020)
		\item Laurent NAVORET % Dernier expose: Modèle macroscopique pour un système de particules discoïdales en interactions d'alignement (2015)
  \end{itemize}
  Responsables des seminaires en EDP

  \pause
  % L'equipe MOCO en general (analyse des EDP, de la théorie du contrôle, du calcul scientifique et haute performance, et des statistiques.)
	\begin{itemize}
		\item Partenariats internationaux (Portugal, Allemagne, USA, etc.)  % Projets (Examag Spexxa, MAToS, projet EUROFUSION)
		\item Partenariats indutriels  % 
		\item Modélisation des plasmas  % L’équipe projet INRIA TONUS qui lui est adossee
  \end{itemize}

\end{frame}

% \subsection{L'équipe Probabilités}
	
\begin{frame}
  % MM. Franck et Navoret
	L'equipe Probabilités compte plusieurs membres parmi lesquels M.: % M. Prud'homme aussi
	\begin{itemize}
		\item Vincent VIGON
  \end{itemize}

  \pause
  % L'equipe Probabilites en general
  Des activites diverses:
	\begin{itemize}
		\item Partenariats internationaux (Allemagne, Autralie, Chine, etc)  % Actuariat, Transport optimal, Matrrice Aleatoire
		\item Séminaire (de calcul) stochastique.  % 
  \end{itemize}

% You get the point, ce sint de grosses equipes de recherches tres actives! Et des que j'ai vu qu'elles allait encadrer le projet, j'ai saute sur l'occasion

\end{frame}

\subsection{Le sujet du stage}

\begin{frame}

\begin{columns}
 \begin{column}{0.5\textwidth}
  \centering
    Probleme direct \\ (\scriptsize Resolution de l'EDP du transfer radiatif)
    % Image de densite -> signal sur les bords
 \end{column}
 \pause
 \begin{column}{0.5\textwidth}
    \centering
    Probleme inverse \\ (\scriptsize Reconstruction de la densite par un ANN)
    %Image de signal sur les bords -> densite
 \end{column}
\end{columns}

\end{frame}

%-------- Vrai debut de l'introduction (PB INVERSE)
\begin{frame}
  Trois point cles pour situer le stage:
  \begin{enumerate}
    \item Explosion du deep learning % Depuis le debut de la decenie 2010, le Machine Learning a considerablement pris de l’ampleur (2015 a l’ILSVRC, etc..)
    %%%%%% IMAGE DU DEEP LEARNING
    \item APplications dans le secteur medical (Imagerie medicale) % Avant de soigner les cancers, on doit detecter les tumeurs sont plus denses que les tissus sains (Chercher d'autres applications)
    %%%%%% IMAGE DU MEDICAL
    \item Reevaluation des methode de resolution de problemes inverse % Les problemes inverses sont difficiles. ... Les algo d'optimisation classiques marchent tres bien. En fait on s'est referer aux travaux de Maya et Guillaume Dolle. L'avantage que peuvent offrir les ANN c'est juste la simplicite, et la rapidite, et une generalisation (non specificite aux probleme)
    %%%%%% IMAGE DU PB INVERSE
  \end{enumerate}
  
\end{frame}

\begin{frame}
  \scriptsize
  \frametitle{Sommaire}
  \tableofcontents
\end{frame}

%-------------------------------------------------------------------------------
%							SECOND SECTION
%-------------------------------------------------------------------------------
\section{Principe}

\subsection{Simulation de l'ETR}

\begin{frame}
  \frametitle{Le transfer radiatif}

Lorsque la photons se trouvent en presence de la matière, Trois phenomènes majeures (caratises par leurs opacites) se produisent:

\begin{columns}
  \begin{column}{0.5\textwidth}
    \footnotesize
   \begin{itemize}
     \item Emission ($\sigma_e$): Plus la temperature matiere est elevee, plus l'emission est importante % Typiquement on ne vas pas retrouver sigma_e dans nos equations car on va se placer dans l'ETL, et on Planck.
     \item Absorption ($\sigma_a$): Lorsqu'on est a l'equilibre thermique, $\sigma_a = \sigma_e$ % On va considerer en plus l'equilibre chimique ce qui donne l'ETL
     \item Scattering ($\sigma_c$): Il faut aussi tenir compte de la fonction de distribution angulaire de « scattering » $p(\bm{\Omega^\prime \rightarrow \bm{\Omega}})$.
   \end{itemize}
  \end{column}
  % \pause
  \begin{column}{0.5\textwidth}
     \begin{center}
      \includegraphics[width=6cm]{TransferRadiatif}       
     \end{center}
  \end{column}
 \end{columns}
 
\end{frame}

\begin{frame}
  \frametitle{L'ETR}
  L'equation du transfert radiatif est bilan d'energie lie au rayonnement au niveau mesoscopique. 

  \begingroup
  \scriptsize
  \begin{gather*}
      \begin{aligned}
      \frac{1}{c} \frac{\partial}{\partial t}I(t,\bvec{x},\bm{\Omega},\nu) &+\bm{\Omega}\cdot\nabla_{\bvec{x}} I(t,\bvec{x},\bm{\Omega},\nu) \\
      &= \sigma_a(\rho,\bm{\Omega},\nu)\left(B(\nu,T)-I(t,\bvec{x},\bm{\Omega},\nu)\right) \\
      &+ \frac{1}{4\pi} \int_{0}^{\infty} \int_{S^2}\sigma_c(\rho,\bm{\Omega},\nu)p(\bm{\Omega}^\prime\rightarrow\bm{\Omega})\left(I(t,\bvec{x},\bm{\Omega}^\prime,\nu)-I(t,\bvec{x},\bm{\Omega},\nu)\right) \, d\bm{\Omega}^\prime \, d\nu
      \end{aligned}
  % \label{eqn:ETR}
  \end{gather*}
  \endgroup
Où 
\begin{itemize}
  \item $I(t,\bvec{x},\bm{\Omega},\nu)$ designe l'intensité radiative specifique;
  \item $B(\nu,T)$ la fonction de Planck;
  \item $\oint p(\bm{\Omega}^\prime\rightarrow\bm{\Omega})\, d\bm{\Omega}^\prime=1$
\end{itemize}

\end{frame}

\begin{frame}
  \frametitle{Le modele P1}
  % Le modele P1: % Modèle macroscopique 5 aux moments (d’ordre 2), linéaire et hyperbolique. Le terme 1/3
  \begingroup
  \large
  \begin{equation*}
      \begin{cases}
       \partial_tE + c \ \operatorname {div} \bvec F = c\sigma_a\left(aT^4-E\right)\\
       \partial_t\bvec{F} + c \ \nabla E = -c\sigma_c \bvec{F} \\
       \rho C_v \partial_t T = c \sigma_a \left(E-aT^4\right)
      \end{cases}
  % \label{eqn:P1}
  \end{equation*}
  \endgroup
Ou:   % Moins precis que Monte-Carlo ou ordonne discretes; Mais plus rapide et suddisant; sigma
\begin{align*}
  E(t,\bvec{x}) &= \frac{4\pi}{c} \int_{0}^{\infty} \int_{S^2} I(t,\bvec{x},\bm{\Omega},\nu) \, d\bm{\Omega} \, d\nu \\
  \bvec{F}(t,\bvec{x}) &= \frac{4\pi}{c} \int_{0}^{\infty} \int_{S^2} \bm{\Omega}I(t,\bvec{x},\bm{\Omega},\nu) \, d\bm{\Omega} \, d\nu 
  % \label{eqn:EFT}
\end{align*}

\end{frame}

\begin{frame}
  \frametitle{Le schema de « splitting »: Etape 1}
  % Reglage de la temperature 
  \begin{columns}
    \begin{column}{0.5\textwidth}
      On pose $\Theta = aT^4$

      \begingroup
      \normalsize
      \begin{equation*} 
        \begin{dcases}
         E_j^{q+1} = \dfrac{\alpha E_j^n + \beta \gamma \Theta_j^n}{1 - \beta \delta} \\
         \Theta_j^{q+1} = \dfrac{\gamma \Theta_j^n + \alpha \delta E_j^n}{1 - \beta \delta} 
        \end{dcases}
    \label{eqn:Step1}
    \end{equation*}
      \endgroup
      En posant
      \scriptsize
      $\mu_q = \dfrac{1}{T^{3,n} + T^{n}T^{2,q} + T^{q}T^{2,n} + T^{3,q}}$
      \normalsize
      On a
    \end{column}
    % \pause
    \begin{column}{0.5\textwidth}
       \begin{center}
        \includegraphics[width=4.5cm]{Dicretisation2D}       
       \end{center}
    \end{column}
   \end{columns}
   \tiny
  $\quad  \alpha = \dfrac{1}{\Delta t \left( \frac{1}{\Delta t} + c \sigma_a \right)} ,\quad 
   \beta = \dfrac{c \sigma_a}{\frac{1}{\Delta t} + c \sigma_a} ,\quad 
   \gamma = \dfrac{\rho_j C_v \mu_q}{\Delta t \left( \frac{\rho_j C_v \mu_q}{\Delta t} + c \sigma_a \right)} \quad \text{et} \quad  
   \delta = \dfrac{c \sigma_a}{\frac{\rho_j C_v \mu_q}{\Delta t} + c \sigma_a}.$

   \normalsize
   COnvergence ver $E_j^*$ et $\Theta_j^*$. $\bvec F_j$ reste constant egale a $F_j^*$.
   
\end{frame}


\begin{frame}
  \frametitle{Le schema de « splitting »: Etape 2}   % Adaptable aussi en 2D
  \begin{columns}
    \begin{column}{0.6\textwidth}
      \begingroup
      \normalsize
      \begin{equation*} 
          \begin{dcases}
          E_j^{n+1} = E_j^* + \alpha \sum_k \left( \bvec F_{jk}, \bvec n_{jk} \right) \\
          \bvec F_j^{n+1} = \beta \bvec F_j^* + \bm{\gamma} E_j^n + \delta \sum_k E_{jk} \bvec n_{jk}
          \end{dcases}   
      % \label{eqn:Step2}
      \end{equation*}
      Avec :
      % \begingroup
      \scriptsize
    
      % \begin{gather*}    
      % \begin{aligned} 
        $\alpha = -\frac{c \Delta t}{\left| \Omega_j \right|}, \linebreak
        \beta = \frac{1}{\Delta t} \left( \frac{1}{\Delta t} + c \sum_k M_{jk} \sigma_{jk} \right)^{-1}, \linebreak
        \bm{\gamma} = \frac{c}{\left| \Omega_j \right|} \left( \frac{1}{\Delta t} + c \sum_k M_{jk} \sigma_{jk} \right)^{-1} \left( \sum_k l_{jk} M_{jk} \bvec n_{jk} \right) \linebreak
        \delta = -\frac{c}{\left| \Omega_j \right|} \left( \frac{1}{\Delta t} + c \sum_k M_{jk} \sigma_{jk} \right)^{-1}$
    %   \end{aligned}
    % \end{gather*}

    \endgroup
      
    \end{column}
    % \pause
    \begin{column}{0.4\textwidth}
       \begin{center}
        \includegraphics[width=5cm]{Interaction2D}       
        \begingroup
        \tiny
        \begin{align*}
          \left(\bvec F_{jk}, \bvec n_{jk} \right) &= l_{jk} M_{jk} \left( \frac{\bvec F_j^n \cdot \bvec n_{jk} + \bvec F_k^n \cdot \bvec n_{jk}}{2} - \frac{E_k^n - E_j^n}{2} \right) \\
          E_{jk} \bvec n_{jk} &= l_{jk} M_{jk} \left( \frac{E_j^n + E_k^n}{2} - \frac{\bvec F_k^n \cdot \bvec n_{jk} - \bvec F_j^n \cdot \bvec n_{jk}}{2} \right) \bvec n_{jk} \\
        %  \end{align*}
        %  \begin{align*}
          M_{jk} &= \frac{2}{2 + \Delta x \sigma_{jk}}  \\
          \sigma_{jk} &= \frac{1}{2} \left( \sigma_c(\rho_j,T_j^n) + \sigma_c(\rho_k,T_k^n) \right)
         \end{align*}
        \endgroup
       \end{center}
    \end{column}
   \end{columns}
\end{frame}

\subsection{L'architecture de reseau de neurones}

\begin{frame}
  \frametitle{Implementation C++}
  \begin{columns}
    \begin{column}{0.6\textwidth}
      \scriptsize
      \begin{itemize}
        \item Temps final = 0.01 \si{sh} %\textit{(1 shake (\si{sh}) = $10^{-8}$ secondes)}
        \item $c = 299$ [\si{\cm \per sh}]
        \item $a = 0.01372$ [\si{g \per cm \per sh^2  \per keV }]
        \item $C_v = 0.14361$ [\si{Jerk \per\g \per keV}] % \textit{(1 \si{Jerk} = 1\si{m \per \s\cubed})}
        \item La densité $\rho$ est un signal créneau [\si{\g\per\cm\cubed}]
        \item $\sigma_a = \rho T$ [\si{\per\cm}]
        \item $\sigma_c = \rho T$ [\si{\per\cm}]
        \item $T_0, T_{gauche} = 5$ [\si{keV}] % \textit{(en termes de température, 1 \si{keV} = 11605 \si{K})}
        \item $E_0 = aT_0^4$ [\si{g \per \cm \per sh^2}]
        \item $E_{gauche^*} = aT_{0}^4 + 5 \sin (2 k \pi t)$ [\si{g \per \cm \per sh^2}]
        \item $\bvec{F}_0, \bvec{F}_{gauche} = \bvec{0}$ [\si{g \per sh^2}]
        \item Sorties libres sur les autres bords
      \end{itemize}
    \end{column}
    % \pause
    \begin{column}{0.4\textwidth}
      %  \begin{center}
        \includegraphics[width=4.5cm]{SimuCFG}   % EN 2D    
      %  \end{center}
    \end{column}
   \end{columns}
\end{frame}

%-------------------------------------------------------------------------------
%							THIRD SECTION
%-------------------------------------------------------------------------------
\section{Le probleme en 1D}

\subsection{Simulation}

\subsection{Apprentissage}

%-------------------------------------------------------------------------------
%							FOURTH SECTION
%-------------------------------------------------------------------------------
\section{Le probleme en 2D}

\subsection{Simulation}

\subsection{Apprentissage}





% % Chapter 1

\chapter{Introduction} % Main chapter title

\label{Chapter1} % For referencing the chapter elsewhere, use \ref{Chapter1} 

%----------------------------------------------------------------------------------------

% Define some commands to keep the formatting separated from the content 
\newcommand{\keyword}[1]{\textbf{#1}}
\newcommand{\tabhead}[1]{\textbf{#1}}
\newcommand{\code}[1]{\texttt{#1}}
\newcommand{\file}[1]{\texttt{\bfseries#1}}
\newcommand{\option}[1]{\texttt{\itshape#1}}

%----------------------------------------------------------------------------------------

En 2015, le réseau de neurones vainqueur de l'ILSVRC \footnote{ImageNet Large Scale Visual Recognition Challenge} obtient une précision de 97.3 \%, ce qui conduit les chercheurs à affirmer que les machines peuvent identifier les objets dans des images mieux que les humains \parencite{Reference1}. Depuis lors, le domaine du Machine Learning a continué à prendre de l'ampleur. Aujourd'hui ses applications se multiplient dans plusieurs secteurs d'activité parmi lesquelles l'automobile, la finance, le divertissement, et plus important, celui de la santé à travers l'imagerie médicale.

Les tumeurs ont des propriétés optiques différentes des tissus qui les entourent \footnote{Les tumeurs sont généralement plus denses que les tissus sains.}. Étant donné un domaine avec un faisceau lumineux qui s'y propage, reconstruire sa densité à l'aide du signal temporel mesuré sur ses bords constitue un problème inverse. Les problèmes inverses sont très importants en sciences mathématiques et ont des applications variées en imagerie médicale, radar, vision, etc. Ils sont malheureusement très difficiles à résoudre et nécessitent des algorithmes d'optimisation très avancés. Les réseaux de neurones artificiels se présentent comme une méthode potentiellement moins coûteuse mais plus rapide.

Grace à son unité mixte de recherche IRMA, l'UFR de mathématiques et d'informatique de l'Université de Strasbourg est un pôle de recherche en mathématiques appliquées. À travers ses équipes MOCO et Probabilités, l'IRMA s'intéresse aux interactions entre les EDP et le Machine Learning, raison pour laquelle j'ai choisi d'y effectuer mon stage de master 1 CSMI\footnote{Calcul Scientifique et Mathématiques de l'Information}. Au cours de ce stage (du 15 juin au 15 août 2020), j'ai pu m'intéresser au problème inverse de reconstruction de la densité d'un domaine par un réseau de neurones convolutif (CNN).

Ce stage a été suivi par les enseignants-chercheurs MM. Emmanuel \textsc{Franck}, Laurent \textsc{Navoret}, et Vincent \textsc{Vigon} et s'inscrit dans la continuation d'un projet (encadré par la même équipe) qui s'est déroulé du 19 mars au 28 mai 2020. Le projet consistait en la simulation 1D d'un schéma de « splitting » pour le modèle P1 de l'équation du transfert radiatif couplé avec la matière. Le stage quant à lui a essentiellement consisté en la simulation du même schéma en 2D, et en la reconstruction de la densité du milieu par un CNN. Ce stage a été l'opportunité pour moi d'apprendre sur les EDP et sur l'apprentissage profond tout en me familiarisant avec l'interface de programmation de la librairie de réseaux de neurones Keras. 

En vue de rendre compte de manière fidèle des deux mois passés au sein de l'IRMA, il apparait logique de présenter en titre de préambule le cadre du stage et son environnement technique. Ensuite il s'agira de présenter les différentes missions et tâches qui j'ai pu effectuer. Enfin je présenterais un bilan du stage, en incluant les différents apports et enseignements que j'ai pu en tirer.

% % Chapter 2

\chapter{Présentation de l'IRMA} % 2 nd chapter title

\label{Chapter2} % For referencing the chapter elsewhere, use \ref{Chapter2} 

\textit{Les informations presentes dans cette section sont entierement issues du \href{http://irma.math.unistra.fr/}{site web de l'IRMA}}

Cree en 1966 par Jean Frenjel et Gearges Reeeb, l'IRMA\footnote{Institut de Recherche Mathématique Avancée} est une  unité mixte de recherche (UMR 7501) sous la double tutelle du CNRS (a travers l'INSMI\footnote{Institut National des Sciences Mathématiques et de leurs Interactions}) et de l’Université de Strasbourg (UFR de Mathématique et d’Informatique).

Dirigee par le professeur Philippe Helluy, l'IRMA comporte environ 130 membres. On y compte enviro 87 chercheurs et enseignants-chercheurs permanent et une qurantaine de non permanents repartis en 8 equipes de recherche.

Les activites majeures de l'organisme sont l'organisation des seminaires, des journees, des colloques et des conferences. Ces activites sont renforcees par les nombreux partenariats qu'elle maintient dans les secteurs academique(Cemosis, Labex IRMIA, etc..) et indutriel (AxesSIM, Electis, etc..).

%----------------------------------------------------------------------------------------

\section{Structure de l'organisation}
L'organigramme de l'entreprise representant ses sections majeures est donne a la figure \ref{fig:OrganigrammeIRMA}).

\begin{figure}[!h]
\centering
\includegraphics[width=.8\linewidth]{OrganigrammeIRMA} 
\decoRule
\caption[Organigramme de l'IRMA]{Organigramme representant l'organisaiton de l'IRMA au mois de mars 2020 \parencite{Reference7}}
\label{fig:OrganigrammeIRMA}
\end{figure}

%----------------------------------------------------------------------------------------

\section{Les équipes MOCO et Probabilite}

L'equipe MOCO \footnote{MOdelisation et COntrole} se compose de specilistes des EDP, de la theorie deu controle, ddu calcul scientifique et haute performance et des statistiques. Ses activites s'etendent a l'internationale et dans l'indutriel (REF, ...). Les enseignants-chercehurs MM. Emmanuel Franck et en Laurent Navoret y sont responsables des seminaires en equations aux derivees partielles. 

L'equipe probabilite est composee d'experts en calcul de probabilite. Ses membres se retrouvent regulierement lors du "seminaire stochatique". A cette equipe apartient M. Vincent Vigon. 

Je tiens une fois de plus a remercier les trois chercheurs mentiones ci-hauts qui ont encadrer ce stage. La combinaison de ces deux equipes dont ils font partie a permis de faire face aux deux aspects de ce stage. Premierement la modelisation d'EDP et finalement l'utilisation des reseaux de neurones.

%----------------------------------------------------------------------------------------

% % Chapter 3

\chapter{Modelisation de l'EDP en 2D} % 3rd chapter title

\label{Chapter3} % For referencing the chapter elsewhere, use \ref{Chapter3} 

%----------------------------------------------------------------------------------------

Ayant resolu le modele en 1D durant le stage, on procede dans cette partie a sa modelisation en 2D. Il s'agit de resoudre le probleme direct du transfer radiatif avant de passer au probleme inverse dans la partie suivante. On rappelle breivement le modele considere avant de decrite l'implementation que utilisee.

\section{Le transfert radiatif}


On considère un rayonnement transporté par des particules de masse nulle appelés photons. Lorsqu'ils se touvent en presence de la matiere, les photons inteassgissent avec celle. Trois phonomees sont preponderant (\ref{Fig:TranfertRadiatif}):

\begin{itemize}
 \item l'emission: Les photons sont emis en reponse aux electrons excites descendants a des niveaux d'energie plus bas. Ce phenomenes est caraterise par l'opacite d'émission $\sigma_e$. Il s'agit de l'inverse du libre parcours d'emision \footnote{Le libre par cours moyen d'emission represente la distance moyenne entre deux emissions de photons. Les libres parcours d'absorption et de dispersion sont definis de maniere similaire}. Plus la temperature matiere est elevee, plus ce phenomene est important.
 \item l'absorption: A l'inverse, certains photons sont absorbes par la matiere. Ce pehenomene se caraterise par l'opacite d'absorption $\sigma_a$. Lorsqu'on est a l'equilibre thermique, $\sigma_e = \sigma_e$.
 \item la dispersion (ou "scaterring" ou parfois diffusion): Certains photons sont devies de leur trajectoire par la matiere. Ce phenomme se caraterise non seulement par son opacite de scatering $\sigma_c$ \footnote{ $\sigma_a$ et $\sigma_c$ sont definis de maniere similaire a $\sigma_e$}, mais aussi par une fonction de distribution angulaire decrivant la maniere dont les photons sont devies.
\end{itemize}

(IMAGE DU TRANFER RADIATIF)

L'equation du transfer radiatif (1) represente un bilan d'energie lie au rayonnement au niveau microscopique. Nous nous placerons dans le cas particulier d'equilibre thermodynamique local (ETL)\footnote{etat dans lequel on peut definir une temperature pour chaque point du domaine, et l'emission est decrite par la fonction de Planck \parencite{Reference3}}. L'equilibre radiatif \footnote{il se produit si la matiere est a l'equilibre avec le reyonnement. Si on est dans l'ETL, les photons sont emis suivant la fonction de Planck a la temperature de la matiere} quant a lui sera considere comme condition initiale pour les simulations.

(EQUATION 1) - ETR (et definition des termes)

Il est possible de modeliser l'ETR a travers plusieurs modeles. Le modele P1 est un modele macroscopique \footnote{ils ne prennent en compte que les varaibles d'espace et de temps et spm obtenu par integration des termes microscopique tels que I par rapport a la frequence et la direction} aux moments (d'ordre 2), lineaire et hyperbolique. Vu que l'energie du rayonnement n'est pas convervee durant sont interaction avec la matiere, il faut coupler le modele P1 avec une equation regissant l'energie de la matiere. On utilisera une equation d'energie amtiere simplifiee qui ne tient compte que des termes d'echange avec le rayonnement. Le modele P1 couple a la matiere (REF FRACNK) est presente ci-bas:

(EQUATION 2) - MODELE P1  E, F, et T)

Comme on peut le voir a travers la definition de $E$ et $F$, notre modele est dit "gris" car nous l'integrons sur tout le spectre de frequence. En effet, nous ne nous interressons qu'au rayonnement a travers son bilan d'energie transporte par le flux radiatif. Sur ce point, la version du modele P1 que nous avons utilise est mois precise qu'un modele microscopique base soit sur une methode Monte-Carlo ou une methode des ordonnes discrete. Neanmoins notre modele presente l'avantage d'etre tres peu couteux et relativement facile a implementer \parencite{Reference3}. 
%----------------------------------------------------------------------------------------

\section{Schéma de splitting}

Le modele P1 tend vers une equation de diffusion lorsque les opcites d'absorption ($sigma_a$) et de dispersion ($sigma_c$) sont elevees (de facon a ce que $c/sigma_a = 1$). Les schema classiques tels que le schema de Rusanov ne sont pas assez precis pour capturer cette propriete. 

(EQ 3 - LIMITE DE FISSUSION)

Le schema en 2 etape (ou de Splitting) propose par (franck) est assez precis pour traduire la limite de diffusion. Les deux etapes sont resumees co-bas.


\subsection{Etape 1}
La premiere etape (dite etape de couplage ou d'equilibre, ou etape de relaxation de la temperature) permet de regler la temperature sur chaque maille (independament des autres mailles). On ne considere que les equations ou la temperature est impiquee (equations 1 et 3 du modele p1), en fixant la valeur du flux sur chaque maille. Il s'agit d'une methode de point fixe qui est toujorus definie. \parencite{Reference2}

Le domaine rectangulaire est suppose discretise en $N \times M$ mailles uniformes. $j$ denote l'identifiant d'une cellule (Figure \ref{fig:2DMesh}).

(SCHEMA DE SPLiTTING - etape 1)

On itere sur q jusq'a ce que E et $\Theta$ convergent vers $E^*$ et $\Theta^*$.

\subsection{Etape 2}
Il s'agit ici de resoudre les deux EDP hyperboliques en 1 et 2. 
Avasnt d'qttquer le schema schema de splitting, on note que les equations 1 et 2 du modele P1 sont hyperboliques et que La methode des volumes finis est donc adaptee pour les resoudre.

(VOLUMES FINIS)

Nous retournons donc sur le maillage disretise en definisant les different flux numeriques impliques. Durant cette etape, il faut considerer l'ajout de mailles fantommes, ce qui porte le nombre total de volumes a $(N+2) \times (M+2)$.

\begin{figure}[H]
\begin{subfigure}{.6\textwidth}
  \centering
  \includegraphics[width=.8\linewidth]{Dicretisation2D}  
  \caption{Dicretisation du domaine}
  \label{fig:Discretisation2D}
\end{subfigure}
\begin{subfigure}{.4\textwidth}
  \centering
  \includegraphics[width=.8\linewidth]{Interaction2D}  
  \caption{Interaction entre deux mailles j et k}
  \label{fig:Interaction2D}
\end{subfigure}
\caption{Dicretisation du maillage 2D. Sur la figure (A), on peut observer les mailles dites "fantomes" hachurees en rouge. Les quatre maille voisine d'une maille j sont indiquees en vert. Le volume de la maille j est definie par $\Omega_j$. Le nombre de mailles suivant l'rizontale $M$ est choisi telle que le maillage soir uniforme i.e $\Delta x = \frac{x_{max}-x_{min}}{N} = \frac{y_{max}-y_{min}}{M} = \Delta y$. Sur la figure (B), on observe la defintion de la normale sortante $n_{jk}$ de la mailles j. On peut aussi observer la longeur caracteristique $l_{jk}$}
\label{fig:2DMesh}
\end{figure}

(SCHEMA DE SPLiTTING etape 2) (\parencite{Reference4})

%----------------------------------------------------------------------------------------

\section{Implementation en C++}

Le dode de calcul a ete develope durant la 4 eme semaine du stage. Etant donne des parametre du probleme, il permet d'exporter les signaux temporels $E, F \text{et} T$ sur les quatres bords du domaine. Pour raisons de visualisation, il permet aussi dd'exporter les signaux sur l'entierete du domaine en tout temps. Ces signaux peuvent ensuite etre visualiser sous forme d'une animation a l'aide d'un notebook construit a cet effet.

L'executable se nomme \verb|transfer| et est disponible avec le reste du code sur le repository Github \href{https://github.com/desmond-rn/projet-inverse-2d}{projet-inverse-2d}.

\subsection{Configuration du modele}

L'executable necessite un fichier de Configuration pour s'excuter. Les parametres a definir sont indiques ci-dessous.

(RAPPELLER L'ASPECT D'UN FICHIER CONFIG)

Un exemple de fichier de configuration pourrait se predenter comme ceci:

(IMAGE DF SIMU)


\subsection{Sauvegarde des données}

Comme mentione ci-haut, on dispose de deux options pour sauvegarder les resultats de la simulation:

\begin{itemize}
 \item sous le forme CSV: Ce mode permet une visualisation facile des resultats a l'aide du noteook. Il est tres couteux en espace memoire et necessite la Librairie Pandas pour le lire en forme de dataframe. Cette operation prend une quantite non negligeagle de RAM, ce qui peut nuire a l'usage qu'on veut faire des donnes.
 
 \item sous le format SDS \footnote{source-densite-signal}: Ce format binaire ne sauvegarde que les informations les plus importante de la simualtion. En locurence la source utilisee, la densite du domaine, et les diffetents signaux sur les bords du domaine. Il est particulieremtnt interressant pour generer les donnees necessaires a l'apprentissage. Les details concernatn la ce format sont donnes en annexe \ref{AppendixA}.
\end{itemize}

%----------------------------------------------------------------------------------------

\section{Résultats}

Queslques resultats obtenus sont presentes ici. Les images ci-bas sont obtenus avec le fichier de configuration (IMAGE DF SIMU). La source est une onde sinusoidale placee en $E$ sur la gauche. La densitee en particulier a la forme d'un signal en crenau egale a 0.1 en dehors du crenau et a 10 en dehors. Les opacites d'absorbes sont proportionelles a la densite.

(ENERGIE ET FLUX OBSTACLE CIRCULAIRE - au temps final)

On peut observer une asorption presque totale du signal au niveau du crenau du a la forte valeur des opacite d'absorption et d'emission. En ce sens, le saut de densite agit comme un obstacle a la propagation du signal. l'evolution de l'energie et du flux sur les bords du domaine traduit l'effet qu'a la densite sur la propagation du signal. Sur les images, la cause du probleme direct (la densite) est affichee au centre des figures, et les effets sont presentes aux alentours.

(EVOLUTION SUR LES BORDS)

Nous testons ensuite notre modele sur le cas tres particulier de la limite de diffusion, un attout important du schema de splitting que nous avons implemente. Dans ce cas, les opacites en dehors de l'obstacle sont de l'ordre de $c$. Le bord gauche est continuement chauffe et l'obstacle est un crenau ayany la forme d;un rectangle vu du haut. (On se sert des lignes de niveau pour observer les variations du signal avec plus de precisions)

(ENERGIE ET FLUX OBSTACLE RECTANGULAIRE - LIMITE DE DIFFUSION)

(EVOLUTION SUR LES BORDS)

On confirme effectiment l'effet de diffusion du signal dans le domaine. Nous pouvons a present passer au probleme inverse proprement dit. 

%----------------------------------------------------------------------------------------

% % Chapter 4

\chapter{Apprentissage} % 4th chapter title

\label{Chapter4} % For referencing the chapter elsewhere, use \ref{Chapter4} 

L'objectif de cette section est de reconstruire la densite en connaissant l'energie E, le flux F,et la temperature sur les bords du domaine aux cours du temps. Dans la suite, nous ferons une simplification majeure: la densite est supposee un signal en creanu (la forme du crenau vu du haut pouvant etre non connue). Aisni, reconstruire la densite revien juste a predire la position et la hauteur du crenau. La valeur de la densite en dehors du crenau sera aussi supposee connue. Nous recherchons une fonction $f^{-1}$ invese de $f$(fonction definisssant le probleme direct) telle que $y = f^{-1}(X)$. OU y represente la densite(plus precisement les attribut de son aut de densite), et X la les signaux sur les bords. Mais le caractere naturellement mal pose des problem inverse rend difficile la determination de $f^-1$. On procede donc a une approximation par un reseau de neurnoes artificiel (ANN) de $f^{-1}$ notee $$y = HATf^{-1}(X, \theta)$$, Ou $\theta$ represente les parametres du reseau de neuronnes.

%----------------------------------------------------------------------------------------

\section{Description des entrees/sorties}

\subsection{Les entrees}
Les entree sont onpossee des signaux etmporels E, F, et T. A chaque fois, il faut normaliser avant de les nourir au reseau de neuronnes. 

\subsubsection{en 1D}
En 1D, les entrees ne sont constituees que di signal recuperer sur le bord droit du domaine. La forme d'une example en presentee a la figure ... .
\subsubsection{en 2D}
Une entree 2D contient considerableme plus d'informatiosn. Les 3 signaux E, F et T sur les 4 bords y sont inclus. On y inclu aussi un signal correpondant a l'une des quatre positions de la source dans chacun de ces cas. En gros, une source a la forme suivante:

Des exampesl d'entree ont ete sauvegardee et les details pour els recuperer sont donnes en Anexe.

\subsection{Les sorties}
Comme mentione plus hautm nous avons faitr quelques simplifications sur la nature des sorties.
\subsubsection{en 1D}
Il s'agit uniquement de l'abcisse et de la hauteur du saut de densite.
\subsubsection{en 2D}
Comparer a la 1D, il faut rajouter l'ordonnee du saut de densite.

%----------------------------------------------------------------------------------------

\section{Architecture generale}

Un reseau de neuronnes artificel \footnote{nous y fereons reference dans la suite juste par reseau de neurones} est un systeme computationnel base sur le reseau de neuronnes biologique. L'apprentissage profond\footnote{definition du nombre de couche} permet de resoudre des problemes en Machine Learning\footnote{definition} que les methodes telles que la regression ineaire, etc.. ne peuvent pas. Il reussit cela en introduisant des representations des donnes qui s'exprimes sous forme d'autres representations, plus simples cette fois. Les reseaux profonds en aval\footnote{en oposition a un reseau de neurones recurrent qui reutilisent els resultats des model pour s'ameliorer} (ou MLP) consitituent l'exemples typique en apprentissage profond. Il s'agit juste d'une fonction (composition de differentes fonctions) faisant correspondre une serie d'entree a une serie de sortie $f^{-1} = composition de f1, f2, etc.$.

Un MLP est constitue de plusieurs couche (assimilables aux fonction f1, f2, .. precedentes) apprenant chacune un aspect particulier des donnnees.
\begin{itemize}
 \item une couche d'entree
 \item des couches cachee
 \item une couche de sortie
\end{itemize}


(IMAGE D'UN MLP)

Les reseaux de neurones convolutifs sont une forme de MLP spcialises dans le traitement des donnes qui ont une form de grille. Par exemple des series en temps qui peuvent etres vues comme des grilles 1D prenant des samples a interval de temps regulier \parencite{Reference5}. Ils sont donc particulieremt adaptes a la reconstruction de la densite partant des signaux temporels E, F, et T. L'archiytecture de base (proposee par M. Vigon ) que nous allons utiliser est representee a la figure .... L'architecture sera implemntee sous la livrarie de machine leanring Keras (avec Tensorflow backend) Les differentes couches presentes seront detailles dans la suite. Nous indiquerons aussi en quoi elles sont importantes pour notre apprentissage.

(ARCHITECTURES DE BASE, avec et sans pooling)

%----------------------------------------------------------------------------------------

\section{Les couches utilisées}

\subsection{Les couches de convolution}
la convolution est l'operation fondamentale d'un CNN. Il s'agit d'une operation lineaire qui combine deux signaux pour en extraire un troisieme. En general, une operation de convolution se definit par la formule suivante.
(FORMULE DE CONV 9.1 - s(t) = i*k) (ici i est le ssignal d'entree et k est le noyau de la convolution)

En pratique, les signaux temporels ne sont pas continus, ils sont discretises par interval de temps $\Delta t$. Dans ce contexte, la convolution 1D se definit par la formule:
(FORMULE 9.3)
Cette formule doit aussi etre adaptee en 2D vu que nos inputs sont 2D. La formule devient donc:
(FORMULE 9.4)

L'operation de convolution est commutative grace a l'inversion du noyaux relativement au siganal d'entree. Cette propriete, bien qu'importante d'un point de vu therique, ne prensente pas d'avantages majeure du point de vu computationnel. C'est la raison pour laquelle on dispose de l'operation de cross-correlation qui est convolution sans inversion du noyau.

(CROSS-CORRELAYION 1D et 2D - 9.6)

On remarque aussi que le parcours des indices se fait suivant l'input. Il se trouve que c'est plus direct et rapide ainsi, parcequ'il y a moins de varaition dans la plage de valeurs valides pour n et m. 

Plueieurs libraries de machine leanring implemente implementent la cross-corelation mais l'appellent convolution. C'est le cas de Keras lorsqu'elle utilise le backend Tensorflow \parencite{Reference5}.
(IMAGE D'UNE CONV 1D - en mode valide)
(IMAGE D'UNE CONV 2D)

Sous Keras, les proprietes du couche de convolution sont:
(PROPREITES 1D)
(PROPREITES 2D)


Les CNN apportent trois notions cles a un apprentissage:
\begin{itemize}
 \item l'interaction creuse: contrairemetn aux couches traditionneles, les couches de convolution utilisent des noyaux de taille condiereblemen inferieure a celle de l'input. En terme de multiplication matricielle, cela permet de faire des taches toutes aussi importantes (detection des condouts, floutage, etc..) en ne gardant que peu de parametres en memoire et augmentant l'efficacite statitique. Cela permet aussi de reduire les couts de calcul.
 (IMAGE)
 \item le partage des parametres: les coefficient du noyau de convolution sont reutilises a chaque endroit de la matrice d'entree, contrairement aux couches traditionnelles qui utilise generalement chaque coefficient une seule fois.
 (IMAGE)
 \item la representation equivariante: le partage de parametre introduit la proprite d'equivaraition par translation. SI l'entree change, la sortie change de la meme facon, et le reseau de neurones exploite cela. Par exemple, dans l'etude d'une image, il serait interressant de detecter les contour dans la premiere couche du reseau,vu que ces meme contour sont suceptibles de reaparatire dans la suite. (\parencite{Reference5})
\end{itemize}


Les CNN se prensentent comme un example de principes neuroscientifiques appliques a l'aprentissage machine (NEURSICNCE).En pratigque, presque tous les reseaux de neuronnes convolutifs utilisent une operation appelee "pooling". En effet, dans les architecture de CNN typiques, la couche de convolution est generalement suivi d'une etape dite de detection. Dans cette etape, les resultas lineaires de la convolution sont passes a une fonction non lineaire auniveau d'une couche de d'activaiotn. Nous detaillerons les details de l'activation dans les sections suivantes. Apres cette etape de detection, le pooling est applique pour modifier les resulats encore plus profoncdement.

\subsection{Le MaxPoling}
Une fonction de pooling tranforme les entres voisines par une fonction d'aggregation statistique. Plusieurs fonctions d'aggregations peuvent etres utlisees. Par examples, le maxpooling renvoi le maximum parmis les entrees sur un domaine (rectigne en 1D et rectangulaire en 2D). 

(IMAGE DE MAX POOLING 1D et 2D)

En general, l'operation de pooling permet de rendre la representation approximativemtn invariante aux petite variatons dans l'input. Parlant de l'identifcation d'objects dans une image par exemple, \textit{l'invarianve par translations locales (petites tranalations) peut etre utile si on est plus interresse par la presence de l'object que par sa localisation exacte}\parencite[321ff.]{Reference5}.

Dans le probleme inverse que nous resolvons, on est aimerais non seulement detecter la presence du saut de densite, mais aussi ses coordonnes exactes. Cela nous amenera donc a considerer dans un premier temps une architecture sans pooling, et dans un dexieme temps, avec Pooling.
 
\subsection{Flatten}
L'operation d;applattissage permet de transformer les donnees en quittant de la forme tensorielle (2D avec plusieurs canaux) a une forme vectorielle. Il s'agit en realite d'une etape de preparation a une couche complement connectee.

(IMAGE D'UN FLATTEN)

\subsection{Les couches denses}
Dans cette couche, tous les neurones sont connectes a tous les neurones de la couche precedente. Une couche dense prend les resultats d'une convolution/pooling et en resort des poinds. Les couches de convolution ayant apris des aspects particuliers des donnees, la couche est un moyen facile d'apprndre des combinaisons non lineaires de ces dernieres.

(IMAGE D``UNE COUHE DENSE)

%----------------------------------------------------------------------------------------

\section{Configurations de l'entrainement sous Keras}
Keras proposent une multitude d'otions et d'hyperparatres pour tuner le modele. Les plus importants sont detailles dans les sections suivantes.

\subsection{Les hyper-parametres}

\subsubsection{le taux d'apprentissage}
il s'agit du parametre le plus influant pour notre apprentissage. Il controle a quelle vitesse le modele \footnote{les poids des neurones sont initialises de facon aleatoire} s'adapte au probleme en determinant de quelle quantite les poids des neurones seront mis a jour apres l'agorithme de backpropagation. S'il est tres eleve, il raoidement conduire a solution non optimale; s'il est tres faible, le modele peut reste fige (il fadra alors un nombre eleve d'epoques pour le debloquer).

Avec un taux d'apprentissage egale a 1e-3, nous n'avons ete capable que de detecter la hauteur du crenau en 1D. Cependant il a fallu descendre jusqua 1e-5 pour determiner avec precision l'abcisse, l'ordonne, et la hauteur du crenau en 2D.

\subsubsection{le batch size}
Il s'agit de la taille de chauque paquet de donnnes \footnote{nombre d'instaces d;entrainement selectiones aleatoiremetn} passes au modele durant un epoque. Un batch size faible apporte du bruit au modele vu qu'une partie aleatoire des donnes est tulisee pour mettre ajour les poids des neurones. Ceci permet une meilleure generalisation du modele tout en permettant une limiter la quantite de donnees chargee dans la RAM a chaque epoque.

\subsubsection{le coefficient de reglarisation L2}
Le penalisation permet d;eviter le surapprentissage. (FORMULE DE PENALISATION) Sous Keras, on peut soit penaliser les poids d'une couche (kernel optimizer), ou bien penaliser les resultats de la fonction d'activation de cette couche (activity optimizer). la deuxieme option a offert les meilleures resultats, c'est pourquoi l'avons appliquee aux deux couches de neuronnes denses avec un coefficient de 1e-5.

\subsection{Autres options utilises}

\subsubsection{L'optimiseur}
L'optimization est une methode d'acceleration de l'entrainement. L'optimiseur Adam\footnote{Adaptative moment estimation} combine les proprietes de deux autres algorithmes d'entrainement (AdaGrad et RMSProp).

\subsubsection{Activation RELU}
La fonction d'activation introduit une non-linearite entre les couche. L'avatage majeure de l'activation ReLU\footnote{Rectified Linear Unit} par rapport aux autres fonctions d;activation c'est qu'il n'active pas tous les neurones en meme temps. D'un pooint de vue computationel, elle est tres eficace tout en produidant des resulats satisfaisants.

\subsubsection{Early stopping}

La technique d'early sera notre moyen primaire de lutte contre le sur-appretisage. Nous arreterons l'apprentissage lorsque le score de validation n'aura pas augmente sur 10 epoques.

\subsubsection{Loss MSE}
Pendant la generation des donnees, on a pris soin de pas introduire de donnes aberantes. La MSE qui est plus elevee sur les valeurs aberantes que la MAE est dont plus adaptee ici.
(FORMULE DE MSE)

\subsubsection{Coefficient de determination R2}
Il se definit comme etant le caree du coefficient de correlation entre les predictions et les labels. Alternativement, on peut utiliser la formule suivante:

(FORMULE POUR R2)

On voit que les predictions et les labels sont tres correles sant etre egaux (sans etre egaux), on risque d'avoir un score R2 ce qui n'est pas carateritique des resultats.

\subsubsection{Un score personalise}
On definit donc un nouveau score particulieremtn adapte a nos donnees. On suppose la prediction correcte si elle est suffisament proche du label:
\begin{itemize}
 \item au dizième près pour la position (suivant x ou y)
 \item à l'unité près pour la hauteur
\end{itemize}

%----------------------------------------------------------------------------------------

\section{Resultats}

Nous resumons la sections precedentes en specifiant les paramtres (et leurs noms) utilises tel que mentiones sous Keras.

(TABLEAU 1D 2D de tous les parametres)

Nous avons entraine les architectures en 1D (fig ...) et par la suite en 2D (fig ...) en ajustant les dimensions des coushes connablement.

\subsection{Régression}
% 
    \subsubsection{en 1D}
    On obtient un score R2 de l'ordre de 98\%. On obtient de tres bonnes predictions sur la hateur de l'obstacle qui affecte directement l'amplitude des signaux sur le bord droit du domaine. Ce score n'est pas assez indicatif vu que les predictions sur la position du crenaux ne sont pas assez precises. En effet, le score personalise ne vaut que 26\%.
    
    Ci-dessous sont quelques unes des pires predictions du modele.
    
    (AFFICHER LES PIRES PREDICTIONS 1D)

    Pour remedier a ce probleme de detection de position x d'un obstacle, il faut passer en 2D.  En effet, le probleme inverse est naturellement mal defini dans le sens ou plusieurs entree peuent donner la meme sortie. En 1D, on ne peut mesurer la sortie que sur un seul bord du domaine, ce qui limite beacoup notre aaprntissage.
    
    \subsubsection{en 2D}
    On obtient un score R2 de l'ordre de 98\%. Cette fois le reseau est capable de detecter non seulment la hauteur de l'obstacle, mais aussi son abcisse et son ordonnee. Notre score personnalise nous permet de confirmer cela avec un score (severe) d'environ 92\%.
    
    Ci-dessous sont quelques unes des pires predictions du modele.
    
    (AFFICHER LES PIRES PREDICTIONS 2D)
    
    Le modele se generalise tres bien. En l'utilisant pour predire des obstacles de nature differente, (des rectangles vu du haut au lieu des cercles), le modele s'ensort plutot bien avec des score de .....
    En plus, le modele a prouver etre capable d'apprendre en continu, du moment que les entrees soient toutes normalisee et ayant la meme forme.

\subsection{Classification}
Durant le stage, il a fallu effectuer une classification mutilabel sur les donnes en 2D. QUi permet de placer l'obstacle dans une categorie definie a partir de la source. La classification petmet de detecter juste l;ordonne de l'obstacle. L'image ci-dessous decrit mieux cette classification:

(IMAGE DEFINITION DE LA CLASSIF)

Les donnnes utilisee pour la classification ont une shape differente des autre. On a moins d'iterations en temps mais mais un maille beacoup plus fin (90x90).

Le modele performe relativement bien sur ces taches de classification avec un score que j'ai appele multilabel accuracy (hard) de .. \%. 
Quelques undes de pires prediction sont les suivantes.

%----------------------------------------------------------------------------------------

% % \documentclass{beamer}
% \usetheme{Szeged}

% \begin{document}

%-------------------------------------------------------------------------------
%							FITH SECTION
%-------------------------------------------------------------------------------


\section{Conclusion}
% Conclusion

\subsection{Sur l'apprentissage}
\begin{frame}
    \frametitle{Bilan de l'apprentissage}
    \begin{enumerate}
        \item \textbf{Régression 1D} : Permet de détecter la hauteur du créneau %sur la densité d'un domaine 1D (avec la meilleure corrélation de tous les apprentissages). Elle n'a cependant pas été capable de détecter la position du créneau, probablement dû au caractère mal posé du problème inverse.
        \item \textbf{Classification 2D} : Permet de localiser l'ordonnée du créneau %en le situant par rapport aux sources sur la gauche d'un domaine 2D. En augmentant leur nombre et en plaçant certaines sources en haut (ou en bas) du domaine, on pourrait localiser avec plus de finesse l'abscisse et l'ordonnée du créneau.
        \item \textbf{Régression 2D} : Permet de prédire tous les attributs essentiels du créneau (abscisse, ordonnée, et hauteur)%, tout ceci avec une très forte précision (score personnalisé s'élevant à 93 \%). 
      \end{enumerate}
\end{frame}

\subsection{Generale}
\begin{frame}
    \frametitle{Bilan du stage}
    \begin{figure}
        \includegraphics[width=10cm]{MilestonesRoadmap}       
        \caption{Points tournants du le stage}
    \end{figure}
\end{frame}

\begin{frame}
    \frametitle{Apports et enseignements}
    \begin{itemize}
        \item Developpement C++ et Python   % Mettre en pratique les connaissances de CSMI
        \item Equations aux derivees partielles % technique de verification
        \item Reseaux de neurones % (Keras, Tensorflow, learning rate)
        \item Experience dans un milieu de recherche % J'ai apprecieer travailler sur IA+EDP
        % \item Point negatif: Manque de coordination (A cause du COVID)
    \end{itemize}
\end{frame}


% \end{document}


% Chapter 5

\chapter{Bilan du stage} % 5th chapter title

\label{Chapter5} % For referencing the chapter elsewhere, use \ref{Chapter5} 

%----------------------------------------------------------------------------------------

\section{Ressources utilisées}

Les ressources utilisee durant le stage varient en nature et en fonction.

\subsection{Ecriture du Code}
\begin{itemize}
 \item VSCode: Pour l'edition du (principalemtn C++) grace a ses fonction. L'extension majeure ici est CMake.
 \item Google Colab: Facilitation de l'apprentissage sous Keras grace a ses GPU. Les libraries majeures ici sont Numpy, pandans, et Keras.
 \item Jupyter: Pour les taches en Python ne necessitant pas trop de resource (visualisation, sauvegarde en format PQT). Les librairies majeures utilisees ici sont Numpy, Pandas et Matplotlib.
 \item Kile: Pour l'ecritude du rapport en Latex
 \item Draw.io: Pour les illustrations 
\end{itemize}


\subsection{Communication}
Les communications se sont effectuees principalemtn par messagerie electronique. j'ai aussi eu l'occasion de communiquer avec les proffeseurs en presentiel a 3 reprise.

%----------------------------------------------------------------------------------------

\section{Journal de bord}
\label{sec:Journal}

\subsection{Semaine 1 et 2}
\begin{itemize}
 \item 15 juin: Reunion de debut de Stage par Google Meet
 \item 16 juin: Demande aux professeurs de verifier un example de simulation 1D, avant de me lancer la generation des donnnees
 \item 17 juin: Remarque du problem d'apparition du crenau sur l'energie 
 \item 18 juin: Redaction d'un nouveau schema par M. Franck (pour l'etape 1) qui devrait conserver l'equilibre
 \item 22 juin: Detection de la source du probleme du crenau sur E, et redefinition des termes. 
 \item 23 juin: Confirmation de l'exactitude des simulations 1D et debut de la generation des donnes avec 500 mailles.
\item 25 juin: Demande d'aide a M. Vigon pour la configuration de la fonction d'activation de la couche de sortie
\end{itemize}


\subsection{Semaine 3 et 4}

\begin{itemize}
 \item 3 juillet: Rencontre avec M. Navoret pour discuter des avancements. Prise de connaissance de d'une des raisons potentielles du probleme de mauvaise prediction de la position du crenau sur la densite en 1D. Proposition de plusieurs solutions par M. Navoret, entre autre de partir d'un signal stationanire sinnusoidal et d'introduire l'onde a un temps t*>0.
 \item 6 juillet: Nouvelles simualtions effectues en vue d'observer la difference entre les effets de deux densites differentes. Continuation vers des nouvelles simualtiosn avec 300 mailles.
 \item 8 juillet: decroissance du taux d'apprentissage a la suggestion de M. Franck mais non amelioration des resultats d'apprentissage.
 \item 9 juillet: Passage aux reseaux convolutif grace a M. Vigon
 \item 11 juillet: Plot du debut des oscillation, des maximum, des minimum a la demande de M. Vigon, afin de mieux observer les effet de deux crenaux de densite diferents. 
\end{itemize}

\subsection{Semaine 5 et 6}

\begin{itemize}
 \item 13 juillet: Rencontre avec M. Navoret et M. Franck a la fac. Denvant la persistance du probleme de non detection de la position du crenau, l'implementation du probleme en 2D semble etre la solution approprie.
 \item 14 juillet reformulation 2D du schema de splitting et adaptation du code 1D en 2D
 \item 19 juillet: fin du codage 2D  er presentation des resultats
 \item 25 juillet: ajustemetn de la gamme de couleurs pour les visualisations et passage a la generation des donnes sur 90x90 mailles.
\end{itemize}

\subsection{Semaine 7, 8, et 9}
\begin{itemize}
 \item 5 aout: Rencontre avec M. Franck a la fac. Proposition de solutions pour la non dectection de la position du crenau en 2D par resolution d'un systeme proche de l'eq de la chaleur, apres affichage par ligne de niveau. La possibilite d'adopter un obstacle s'etendant sur toute la verticale est envidagee. Prise de connaissance des delains pour la redaction du rapport.
 \item 6 aout: Redaction et envoi du plan du rapport de stage. 
 \item 7 sout: Proposition de reduction drasque de resolution spatiale par M. Vigonm, et proposition de nouvelles idees par M. Vigon, entre autre la consideration d'un obstacle considerableme plus opaque.
 \item 8 sout: Nouvel apprentissage avec des simplification majeures qui fonctionnne. Melioration des resultats et continuation du rapport.
 \item 15 aout: Soumission du premier brouillon du rapport
\end{itemize}
%----------------------------------------------------------------------------------------

\section{Difficultés rencontrées et solutions apportées}

Un resume du deroulement du stage est presente a la figure \ref{fig:MilestonesRoadmap}. Les points tounant les plus marquants du stage y sont representes. On peut aussi y voire les diffultes majeures auquelles j'ai ete confronte.

\begin{figure}[H] 
\centering
\includegraphics[width=.8\linewidth]{MilestonesRoadmap} 
\decoRule
\caption[MilestonesRoadmap]{Resumes des etapes marquantes du stage. Les details du deroulement peuvent etre obtenues dans la section \ref{sec:Journal}.}
\label{fig:MilestonesRoadmap}
\end{figure}

\subsection{Apparition d'un crenau sur E, F, et T}
Au commencement du stage, le code de calcul 1D n'etait pas au point. En effet, des qu'on placait un crenau sur la densite, un crenau correpondant se formait puis se propageait sur les signaux E, F, et T. Le probleme a ete resolu par rajout d'un terme au niveau de la deuxieme equation du schema de splitting.

\subsection{Detection de la position du crenau}
La detection de la position du saut de densite a ete un probleme majeure durant le stage. A la fin stage, aucune solution (si elle existe) n'a ete trouvee pour le probleme inverse en 1D. 
Cependant en 2D le probleme a ete resolu essentiellment par augmentation du nombre d'epoques et dimunution du taux d'appretissagee a 1e-5. Il est bien connu que les problemes de machine peuvent diverger si le taux d'apprentissage est trop eleve. Quand au nombre d'eqpoques, je n'en faisait pas suffisament pour voir le modele converger. Une solution bien plus rapide aurait ete d'automatiser la recherche des hyper-parametres, chose que je n'ai apprise qu'a la fin du stage.

Pour revenir a l'apprentissage en 1D, il est important de mentionner quelques pistes d'etude qui, combinee, auraient pu ameliorer les resultats ():
\begin{itemize}
    \item deux apprentissage different: en creant un modele separes pour l'apprentissage de la postion  et de la hauteur
    \item elimination des mauvais labels: on pourrait supprimer de l'appretissage tous les examples qui ont un label trop proche du bord du domaine (par exemple compris entre 0.2 et 0.8 comme en 2D).
\end{itemize}

Cela dit, la solution touvee en 2D est loin d'etre optimale; elle souffre de deux problemes majeures:
\begin{itemize}
 \item pas de generalisation: le modele finalement retenu ne comporte pas de couche de maxpooling
 \item modle trop lourd: le nombre de parametres est tres eleve du a la taille des entree (168, 28, 48) ce qui conduit a un espace memoire tres important au moment de la sauvegarde des poids.
\end{itemize}

Toutes ces quatres pistes d'amelioration n'ont pas ete implemntee pour cause de temps. La gestion de temps a ete un vrai probleme lors du stage.

\subsection{Gestion du temps}
La gestion du temps durant le stage n'a aps ete facile. Au moment d'imimpplementer le schema en 2D (ce qui n'etait pas initialement prevu), j'ai longement hesiter sur l'option la plus rapide. J'ai pu compter sur les conseils de M. Navoret pour surmonter cet obstacle. 

Aussi, je me suis rendu compte des delais bien en retard, j'ai du me debrouiller pour ameliorer les resultat et terminer l'apprentissage. Cela dit, je n'ai pas reussi a faire une partie essentielle qui consiste a verfier comment un modele un modele avec MaxPooling se generalise mieux qu'un modele sans.

%----------------------------------------------------------------------------------------

\section{Les apports du stage}

Ce stage a ete enrichissant pour moi sur plusieurs front:

\subsection{Experience en developpement}
J'ai gagne de l'experience en development C++ et Python, tout en me developant un portfolio. J'ai beaocup apris sur l'API de Pandas, Matplotlib, et plus important encore, celle de Keras. J'ai a present une large base de donnes de code reutilisable pour d'autres taches.

\subsection{Equations aux derivee aprtielles}
J'ai pu observer directemnt quelques astuces utilisees par mes maitres de stages pour verfier la validite de la modelisatopm d'une EDP. Pour l'equation du transfer radiatif, j'ai compris la necessite de partir d'un etat d'equilibre radiatif.

\subsection{Reseau de neurones}
Ce stage m'a permis de percevori la puissance des reseaux de neurones. J'ai appris a quel point le taux d'apprentissage est important. Comme mentionne dans le livre de reference Deep learning \textit{The learning rate is perhaps the most important hyperparameter. If you have time to tune only one hyperparameter, tune the learning rate} \parencite[417]{Reference5}. 

J'en ressort aussi avec quelques question concernant le batch size. Lors de l'apprentissage, il a fallu entrainer le modele en utlisant la methode d'augmentation du batch size pour obtenir les premiers "bons" resultats. Cette methode referencee ici (LiEN RETROUVABLE DANS LES MAIL) montre que beacoupd de questions restent a resoudre dans le domaine du deep learning.

\subsection{Experience de recherche}
En tant que premiere experience dans un environnement de recherche tel que l'UFR, j'ai pu me familirser avec le milieu. J'ai notament apris que les resultats ne doivent pas toujours etre ceux auxquels on s'attends, du moment que l'on a une explication de l'echer.

%----------------------------------------------------------------------------------------


\end{document}
