% Chapter 6

\chapter{Conclusion} % 6th chapter title

\label{Chapter6} % For referencing the chapter elsewhere, use \ref{Chapter6} 

%----------------------------------------------------------------------------------------

Pour conclure, j’ai effectué mon stage de master 1 en tant que stagiaire en calcul scientifique et traitement de données à l'UFR de mathématiques et d'informatique de l'Université de Strasbourg. Lors de ce stage de deux mois, je me suis confronté au problème inverse de reconstruction de la densité d'un domaine par un CNN après avoir simulé la propagation d'un signal dans ce dernier en 2D. J’ai ainsi pu mettre en pratique mes connaissances (en calcul scientifique, en méthodes numériques, et en développement C++ et Python) acquises durant ma formation de CSMI.

En s'intéressant à un problème d'interaction entre l'intelligence artificielle (IA) et les EDP, ce stage m'a permis d'apprendre sur plusieurs plans. J'ai pu approfondir mon savoir sur les notions clés (« backpropagation », descente de gradient, taux d'apprentissage, etc.) utilisées dans les réseaux de neurones profonds. J'ai gagné beaucoup d'expérience en développement logiciel et je me suis familiarisé avec l'environnement de Keras. Ce stage m’a aussi permis de comprendre le déroulement d'une activité de recherche, et à quel point une bonne organisation et un certain degré d'autonomie sont importants. Un point sur lequel j'aurais voulu m'améliorer davantage est celui de la théorie des EDP, en particulier l'intuition derrière la définition des flux numériques. 

Cette expérience de stage fut centrée autour de la problématique de l'apport des réseaux de neurones dans la résolution des problèmes inverses, applicable à la détection des tumeurs \footnote{Les tumeurs sont assimilables à des créneaux, des sauts, ou des obstacles sur la densité du milieu.}. Le modèle de CNN construit est la preuve qu'un réseau de neurones est capables de reconstruire fidèlement des sauts de densité de nature cylindrique (ou circulaire vu du haut). L'utilisation du Max-pooling est sans doute un atout important pour la généralisation du modèle (détection de formes d'obstacles diverses, différentes expressions des opacités, etc.).

Fort de cette expérience et de ses nombreux enjeux, j'aimerais beaucoup par la suite, via un prochain stage, affiner les résultats à l'aide d'un apprentissage en continu en passant à la détection de plusieurs sauts de densités par exemple. On pourrait aussi envisager la reconstruction des opacités d'absorption et de dispersion. Tout ceci pourrais conduite ultimement à la création d'un tomographe et un déploiement en milieu médical.
