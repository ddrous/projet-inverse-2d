% Chapter 5

\chapter{Bilan du stage} % 5th chapter title

\label{Chapter5} % For referencing the chapter elsewhere, use \ref{Chapter5} 

%----------------------------------------------------------------------------------------

\section{Ressources utilisées}

Les ressources utilisee durant le stage varient en nature et en fonction.

\subsection{Ecriture du Code}
\begin{itemize}
 \item VSCode: Pour l'edition du (principalemtn C++) grace a ses fonction 
 \item Google Colab: Facilitation de l'apprentissage sous Keras grace a ses GPU.
 \item Jupyter: Pour les taches en Python ne necessitant pas trop de resource (visualisation, sauvegarde en format PQT)
 \item Kile: Pour l'ecritude du rapport en Latex
 \item Draw.io: Pour les illustrations 
\end{itemize}


\subsection{Communication}
Les communications se sont effectuees principalemtn par messagerie electronique. j'ai aussi eu l'occasion de communiquer avec les proffeseurs en presentiel a 3 reprise.

%----------------------------------------------------------------------------------------

\section{Journal de bord}

\subsection{Semaine 1 et 2}
\begin{itemize}
 \item 15 juin: Reunion de debut de Stage par Google Meet
 \item 16 juin: Demande aux professeurs de verifier un example de simulation 1D, avant de me lancer la generation des donnnees
 \item 17 juin: Remarque du problem d'apparition du crenau sur l'energie 
 \item 18 juin: Redaction d'un nouveau schema par M. Franck (pour l'etape 1) qui devrait conserver l'equilibre
 \item 22 juin: Detection de la source du probleme du crenau sur E, et redefinition des termes. 
 \item 23 juin: Confirmation de l'exactitude des simulations 1D et debut de la generation des donnes avec 500 mailles.
\item 25 juin: Demande d'aide a M. Vigon pour la configuration de la fonction d'activation de la couche de sortie
\end{itemize}


\subsection{Semaine 3 et 4}

\begin{itemize}
 \item 3 juillet: Rencontre avec M. Navoret pour discuter des avancements. Prise de connaissance de d'une des raisons potentielles du probleme de mauvaise prediction de la position du crenau sur la densite en 1D. Proposition de plusieurs solutions par M. Navoret, entre autre de partir d'un signal stationanire sinnusoidal et d'introduire l'onde a un temps t*>0.
 \item 6 juillet: Nouvelles simualtions effectues en vue d'observer la difference entre les effets de deux densites differentes. Continuation vers des nouvelles simualtiosn avec 300 mailles.
 \item 8 juillet: decroissance du taux d'apprentissage a la suggestion de M. Franck mais non amelioration des resultats d'apprentissage.
 \item 9 juillet: Passage aux reseaux convolutif grace a M. Vigon
 \item 11 juillet: Plot du debut des oscillation, des maximum, des minimum a la demande de M. Vigon, afin de mieux observer les effet de deux crenaux de densite diferents. 
\end{itemize}

\subsection{Semaine 5 et 6}

\begin{itemize}
 \item 13 juillet: Rencontre avec M. Navoret et M. Franck a la fac. Denvant la persistance du probleme de non detection de la position du crenau, l'implementation du probleme en 2D est la solution adoptee.
 \item 14 juillet reformulation 2D du schema de splitting et adaptation du code 1D en 2D
 \item 19 juillet: fin du codage 2D  er presentation des resultats
 \item 25 juillet: ajustemetn de la gamme de couleurs pour les visualisations et passage a la generation des donnes sur 90x90 mailles.
\end{itemize}

\subsection{Semaine 7 et 8}
\begin{itemize}
 \item 5 aout: Rencontre avec M. Franck a la fac. Proposition de solutions pour la non dectection de la position du crenau en 2D par resolution d'un systeme proche de l'eq de la chaleur, apres affichage par ligne de niveau. La possibilite d'adopter un obstacle s'etendant sur toute la verticale est envidagee. Prise de connaissance des delains pour la redaction du rapport.
 \item 6 aout: Redaction et envoi du plan du rapport de stage. 
 \item 7 sout: Proposition de reduction drasque de resolution spatiale par M. Vigonm, et proposition de nouvelles idees par M. Vigon, entre autre la consideration d'un obstacle considerableme plus opaque.
 \item 8 sout: Nouvel apprentissage avec des simplification majeures qui fonctionnne. Melioration des resultats et continuation du rapport.
\end{itemize}
%----------------------------------------------------------------------------------------

\section{Difficultés rencontrées et solutions apportées}

\subsection{Apparition d'un crenau sur E}
Au totu debut du stage, un crenau se formait puis se propageait sur l'energie E. Grace a mes encadrant, ce probleme a ete resolu par rajout d'un terme au niveau de la deuxieme equation du schema de splitting.

\subsection{Detection de la position du crenau}
La detection de la position du saut de densite a ete un probleme majeure durant le stage. A la fin stage, aucune solution (si elle existe) n'a ete trouvee pour le probleme inverse en 1D. 
Cependant en 2D le probleme a ete resolu essentiellment par augmentation du nombre d'epoques et dimunution du taux d'appretissagee a 1e-5. Il est bien connu que les problemes de machine peuvent diverger si le taux d'apprentissage est trop eleve. Quand au nombre d'eqpoques, je n'en faisait pas suffisament pour voir le modele converger. Une solution bien plus rapide aurait ete d'automatiser la recherche des hyper-parametres, chose que je n'ai apprise qu'a la fin du stage.


%----------------------------------------------------------------------------------------

\section{Les apports du stage}

Ce stage a ete enrichissant pour moi sur plusieurs front:

\subsection{Experience en developpement}
J'ai gagne de l'experience en development C++ et Python, tout en me developant un portfolio. J'ai beaocup apris sur l'API de Pandas, Matplotlib, et plus important encore, celle de Keras. J'ai a present une large base de donnes de code reutilisable pour d'autres taches.

\subsection{Equations aux derivee aprtielles}
J'ai pu observer directemnt quelques astuces utilisees par mes maitres de stages pour verfier la validite de la modelisatopm d'une EDP. Pour l'equation du transfer radiatif, j'ai compris la necessite de partir d'un etat d'equilibre radiatif.

\subsection{Reseau de neurones}
Ce stage m'a permis de percevori la puissance des reseaux de neurones. J'ai appris a quel point le taux d'apprentissage est important. Comme mentionne dans le livre de reference Deep learning \textit{The learning rate is perhaps the most important hyperparameter. If you have time to tune only one hyperparameter, tune the learning rate} \parencite[417]{Reference5}. 

J'en ressort aussi avec quelques question concernant le batch size. Lors de l'apprentissage, il a fallu entrainer le modele en utlisant la methode d'augmentation du batch size pour obtenir les premiers "bons" resultats. Cette methode referencee ici (LiEN RETROUVABLE DANS LES MAIL) montre que beacoupd de questions restent a resoudre dans le domaine du deep learning.

\subsection{Experience de recherche}
En tant que premiere experience dans un environnement de recherche tel que l'UFR, j'ai pu me familirser avec le milieu. J'ai notament apris que les resultats ne doivent pas toujours etre ceux auxquels on s'attends, du moment que l'on a une explication de l'echer.

%----------------------------------------------------------------------------------------
