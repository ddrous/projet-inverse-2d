
% Appendix B

\chapter{Comment faire des prédictions avec ce modèle?} % Appendix B title

\label{AppendixB} % For referencing this appendix elsewhere, use \ref{AppendixA}
Pour faire des prédications, il suffit de disposer d'un jeu de données cohérent (les entrées par exemple doivent avoir la forme bien particulière décrite à l'une des figures \ref{fig:entrees1D} ou \ref{fig:entrees2D}). Il faut ensuite charger le modèle à l'aide de Keras et ensuite le compiler.

\section{Normalisation des données}
En plus d'avoir la forme appropriée, les données doivent être normalisées i.e toutes les énergies doivent êtres divisées par leur maximum (en valeur absolue). Il en est de même pour le flux et la température.

\section{Chargement du modèle}
Le modèle a été sauvegardé sous la convention HDF5 de Keras. Après l'avoir chargé, il faut le compiler avec l'optimiseur Adam (et un taux d'apprentissage de \verb|1e-4| ou \verb|1e-5| suivant qu'on soit en 1D ou 2D). Pour que la compilation fonctionne, il faut impérativement inclure la fonction de calcul du sore $R^2$ indiquée ci-dessous.

\begin{verbatim}
from keras import backend as K

 def r2_score(y_true, y_pred):
    SS_res =  K.sum(K.square(y_true - y_pred), axis=-1) 
    SS_tot = K.sum(K.square(y_true - K.mean(y_true)), axis=-1)
    return 1.0 - SS_res/(SS_tot + K.epsilon())
\end{verbatim}


\section{Entrainer le modelé en continu}
Le modèle peut être entraîné en continu. Après l'avoir chargé, on peut l'apprendre à détecter d'autres formes d'obstacles sous différentes conditions. Il suffit déjà de disposer de telles données. Pour l'apprentissage en continu, on pourra utiliser la fonction \verb|train_on_batch| de Keras.

