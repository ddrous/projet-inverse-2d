
% Appendix A

\chapter{Comment reproduire les resultats?} % Main appendix title

\label{AppendixA} % For referencing this appendix elsewhere, use \ref{AppendixA}

\section{Execution du code 1D/2D}

Les codes de calculs 1D et 2D se trouvent dans deux repositoy different:
\begin{itemize}
 \item 1D: LIEN
 \item 2D: LIEN
\end{itemize}


Pour compiler le code dans les deux cas, il faut passer par CMake soit directemetn, soit par par un conteneur Docker:
\begin{verbatim}
rm -rf build
cmake -H. -Bbuild
cmake --build build
\end{verbatim}

\section{Lecture du format binaire}
Les donnes peuvent etre sauvegardees au format binaires a l'aide du parametre \verb|export_type binary|. Lorsque c'est la cas, il faut se servir de la fonction ci-dessous en Python pour la lire.

(FONCTION BINARY)

\section{Execution des notebook}
Deux categories de notebooks ont ete crees dans le repository. Les deux premiers sont executables directement apres clonage du repository. Il s'agit de:
\begin{itemize}
 \item Visualisation: Il permet de visualiser les resultats d'une simualtion exportee en CSV dans le fichier \verb|data/df_simu.csv|
 \item Sauvegarde: Il permet de tranformer des donnes du foramt CSV au format binaire PARQUET rapidement lisible par Pandas.
\end{itemize}

Les autres notebook sont executables sur Google Colab. Il ne sont executables que si ont dispose des donnes d;entrainentment, de test et de train. Il faut alternativement telecharger le modele deja entraine. Du a leur tailles condiderable, ces donnes ne sont pas diposbles sur le repository et doivent etre telechargee separement sur (LIEN POUR ), 
