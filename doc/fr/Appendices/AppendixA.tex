
% Appendix A

\chapter{Comment reproduire les resultats?} % Main appendix title

\label{AppendixA} % For referencing this appendix elsewhere, use \ref{AppendixA}

\section{Exécution du code 1D/2D}

Les codes de calculs 1D (développé durant le projet et utilisé pour l'apprentissage durant ce stage) et 2D se trouvent sur deux dépôts Github différents :
\begin{itemize}
 \item 1D: \url{https://github.com/desmond-rn/projet-inverse} 
 \item 2D: \url{https://github.com/desmond-rn/projet-inverse-2d}
\end{itemize}

Pour compiler le code dans les deux cas, il faut idéalement passer par un conteneur Docker. Cependant la compilation peut aussi marcher si on passe par CMake directement.
% \begin{verbatim}
% rm -rf build
% cmake -H. -Bbuild
% cmake --build build
% \end{verbatim}
Ensuite il faut lancer l'exécutable "\verb|transfer|" avec un fichier de configuration. Les détails ainsi que la définition des paramètres 1D/2D du fichier de configuration sont définis dans les fichiers \verb|README.md| des dépôts 1D et 2D respectifs.

\section{Lecture du format binaire}
Pour passer d'une simulation (C++) à sa visualisation ou à l'apprentissage (Python), les données peuvent être sauvegardées au format binaire à l'aide du paramètre "\verb|export_type|". Lorsque c'est le cas, il faut se servir de la fonction \verb|read_sds_version01| (disponible dans le notebook principal de tout l'apprentissage \href{https://colab.research.google.com/drive/18oCXoZzY0_7XnEmHBzHH40vVVnIacoVc?usp=sharing}{Regression.ipynb}) pour les lire.

Les données \textbf{train}, \textbf{val} et \textbf{test} que nous avons utilisé ont été sauvegardées sous le format binaire de Numpy. Ceci permet de faciliter la reproduction de l'apprentissage.


\section{Exécution des notebooks}
Deux catégories de notebooks ont été créés dans les dépôts github. Les deux premiers sont exécutables directement après clonage du dépôts. Ils sont :
\begin{itemize}
 \item Visualisation.ipynb : Il permet de visualiser les résultats d'une simulation exportée en CSV dans le fichier \verb|data/df_simu.csv|.
 \item Sauvegarde.ipynb : Il permet de transformer des données du format CSV au format binaire PARQUET rapidement lisible par Pandas.
\end{itemize}

Les autres notebooks sont exécutables sur Google Colab. Ils ne sont exécutables que si on dispose des données train, val et test. On peut télécharger les modèles 1D/2D déjà entrainés depuis des liens disponibles sur les dépôts Github.
